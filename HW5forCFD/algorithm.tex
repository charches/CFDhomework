\documentclass[12pt, a4paper]{article}
\usepackage{ctex} % 支持中文处理
\usepackage{geometry} % 页面布局
\usepackage{graphicx} % 图片支持
\usepackage{hyperref} % 超链接支持
\usepackage{amsmath} % 数学公式
\usepackage{amsfonts}
\usepackage{amssymb}
\usepackage{amsthm}
\usepackage{bm}
\usepackage{color}
\usepackage{physics}
\newtheorem{lemma}{引理}
\newtheorem{theorem}{定理}
\geometry{left=2.5cm,right=2.5cm,top=2.5cm,bottom=2.5cm} % 设置页边距
\title{数理算法原理}
\author{安庭毅\ 工学院 \ 2100011014}
\date{\today} % 使用今天的日期

\begin{document}

\maketitle % 显示标题
\section{涡量流函数方程推导}
对于如下无量纲化的二维粘性不可压缩流体的方程组:
\begin{align}
    \frac{\partial u}{\partial x} + \frac{\partial v}{\partial y} &= 0 \\
    \frac{\partial u}{\partial t} + u \frac{\partial u}{\partial x} + v \frac{\partial u}{\partial y} &=
    \frac{\partial p}{\partial x} + \frac{1}{Re}\nabla^2u \\
    \frac{\partial v}{\partial t} + u \frac{\partial v}{\partial x} + v \frac{\partial v}{\partial y} &=
    \frac{\partial p}{\partial y} + \frac{1}{Re}\nabla^2v
\end{align}
将$\frac{\partial (3)}{\partial x}$减去$\frac{\partial (2)}{\partial y}$,并引入涡量$\omega=\frac{\partial v}{\partial x}-\frac{\partial u}{\partial y}$和流函数$u=\frac{\partial \psi}{\partial y}, v = -\frac{\partial \psi}{\partial x}$,得到:
\begin{align}
    \frac{\partial \omega}{\partial t} + \frac{\partial \psi}{\partial y}\frac{\partial \omega}{\partial x} - \frac{\partial \psi}{\partial x}\frac{\partial \omega}{\partial y} &= \frac{1}{Re}\nabla^2{\omega} \\
    \nabla^2 \psi &= -\omega \\
    u=\frac{\partial \psi}{\partial y}&, v = -\frac{\partial \psi}{\partial x}
\end{align}
即为涡量流函数方程。

\section{求解步骤及边界条件设置}
对于以上涡量流函数方程,网格离散后按如下步骤求解:

1.计算当前时刻内点处的涡量,空间导数均使用中心差分格式计算;

2.计算当前时刻流函数,这相当于通过sor迭代求解poisson方程;

3.计算当前时刻边界上的涡量。

在本问题中,边界均为流线,流函数在边界上为常值,设为0;涡量在边界上的取值由如下Thom公式计算:
\begin{align}
    \omega_b = -\frac{2(\psi_{inner} - \psi_b + v_\tau h)}{h^2}
\end{align}
\end{document}