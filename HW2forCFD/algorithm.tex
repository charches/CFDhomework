\documentclass[12pt, a4paper]{article}
\usepackage{ctex} % 支持中文处理
\usepackage{geometry} % 页面布局
\usepackage{graphicx} % 图片支持
\usepackage{hyperref} % 超链接支持
\usepackage{amsmath} % 数学公式
\usepackage{amsfonts}
\usepackage{amssymb}
\usepackage{amsthm}
\usepackage{bm}
\usepackage{color}

\geometry{left=2.5cm,right=2.5cm,top=2.5cm,bottom=2.5cm} % 设置页边距
\title{数理算法原理}
\author{安庭毅\ 工学院 \ 2100011014}
\date{\today} % 使用今天的日期

\begin{document}

\maketitle % 显示标题

\section{格式构造与精度}

记步长为$h$,考虑$u$在$j$处网格点的Taylor展开:

\begin{align}
    u_{j-1} &= u_{j} - \frac{\partial{u}}{\partial{x}}h + \frac{\partial^2{u}}{\partial{x}^2}\frac{h^2}{2} - \frac{\partial^3{u}}{\partial{x}^3}\frac{h^3}{6} + O(h^4) \label{eq:j-1}\\
    u_{j+1} &= u_{j} + \frac{\partial{u}}{\partial{x}}h + \frac{\partial^2{u}}{\partial{x}^2}\frac{h^2}{2} + \frac{\partial^3{u}}{\partial{x}^3}\frac{h^3}{6} + O(h^4) \label{eq:j+1}\\
    u_{j+2} &= u_{j} + \frac{\partial{u}}{\partial{x}}2h + \frac{\partial^2{u}}{\partial{x}^2}2h^2 + \frac{\partial^3{u}}{\partial{x}^3}\frac{4}{3}h^3 + O(h^4) \label{eq:j+2}
\end{align}
分别考虑一阶导数和二阶导数的向前差分和中心差分格式。

\subsection{向前差分及精度}
由\eqref{eq:j+1}得:
\begin{align}
    \frac{\partial{u}}{\partial{x}} = \frac{u_{j+1} - u_{j}}{h} + O(h)
\end{align}
故一阶导数的向前差分是一阶精度。

由\eqref{eq:j+2}$ - 2 * $\eqref{eq:j+1},再同时加上$u_{j}$,得
\begin{align}
    \frac{\partial^2{u}}{\partial{x}^2} = \frac{u_{j+2} - 2u_{j+1} + u_{j}}{h^2} + O(h)
\end{align}
故二阶导数的向前差分也是一阶精度。

\subsection{中心差分及精度}
由\eqref{eq:j+1}$-$\eqref{eq:j-1}得:
\begin{align}
    \frac{\partial{u}}{\partial{x}} = \frac{u_{j+1}-u_{j-1}}{2h} + O(h^2)
\end{align}
故一阶导数的中心差分是二阶精度。

由\eqref{eq:j+1}$+$\eqref{eq:j-1}再同时减去$2*u_j$得:
\begin{align}
    \frac{\partial^2{u}}{\partial{x}^2} = \frac{u_{j+1}+u_{j-1}-2u_j}{h^2} + O(h^2)
\end{align}
故二阶导数的中心差分是二阶精度。

\section{数值验证精度}
设精确值为$u_e$,以h为步长得到的计算结果为$u_h$,若精度为p阶,则有:
\begin{align}
    \lim_{h \to 0^+}\frac{\vert{u_e-u_h}\vert}{h^p} = C 
\end{align}
其中C为一大于0的常数。

于是,对于两倍的步长,有:
\begin{align}
    \lim_{h \to 0^+}\frac{\vert{u_e-u_{2h}\vert}}{h^p} = 2^pC
\end{align}

两式相除并取以2为底对数,得
\begin{align}
    \lim_{h \to 0^+}\log_2\frac{\vert{u_e-u_{2h}\vert}}{\vert{u_e-u_h}\vert} = p
\end{align}

因此,通过计算h趋于0时上式的极限,就可判断收敛阶。
\end{document}
