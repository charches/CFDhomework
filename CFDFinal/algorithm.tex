\documentclass[12pt, a4paper]{article}
\usepackage{ctex} % 支持中文处理
\usepackage{geometry} % 页面布局
\usepackage{graphicx} % 图片支持
\usepackage{hyperref} % 超链接支持
\usepackage{amsmath} % 数学公式
\usepackage{amsfonts}
\usepackage{amssymb}
\usepackage{amsthm}
\usepackage{bm}
\usepackage{color}
\usepackage{physics}
\newtheorem{lemma}{引理}
\newtheorem{theorem}{定理}
\geometry{left=2.5cm,right=2.5cm,top=2.5cm,bottom=2.5cm} % 设置页边距
\title{数理算法原理}
\author{安庭毅\ 工学院 \ 2100011014}
\date{\today} % 使用今天的日期

\begin{document}

\maketitle % 显示标题
\section{FVS通量分裂}
在讨论欧拉方程组的半离散差分格式时,若使用迎风格式,
需要根据波传播的特征方向采用不同的基点计算数值通量。因此,在开始计算之前,可以利用FVS方法将通量按一定方式分裂为正负方向。
在本次作业中,我们统一使用Steger-Warming通量分裂方法,其基本思路如下:

通量$F(U)$可以分解为$A(U)U$,其中$A(U)$是通量对守恒变量的Jacobi矩阵。由双曲性,$A$可以对角化,记$A=R\Lambda R^{-1}$。将$\Lambda$分解为正负部分,并以小量
修正保证分裂后矩阵的光滑性,即:
\begin{align}
    \lambda^{\pm} = \frac{\lambda\pm\sqrt{\lambda ^ 2 + \epsilon ^ 2}}{2} 
\end{align}
再计算$F\pm=R\Lambda^{\pm}R^{-1}U$即得到分裂后沿不同方向的通量,分别利用这两个通量按对应的差分格式计算界面数值通量即可。
由于Jacobi矩阵的对角化可以解析算出,所以无需数值求解对角化。对于$\Lambda$是对角线元素为$u, u+c, u-c$处理之后的对角矩阵,对应的通量可由如下公式计算:

\begin{align}
\mathbf{F} = \frac{\rho}{2\gamma}
\begin{bmatrix}
2(\gamma - 1)\lambda_1 + \lambda_2 + \lambda_3 \\
2(\gamma - 1)\lambda_1 u + (u + c)\lambda_2 + (u - c)\lambda_3 \\
(\gamma - 1)\lambda_1 u^2 + \dfrac{3 - \gamma}{2(\gamma - 1)}(\lambda_2 + \lambda_3)c^2 + \dfrac{1}{2}\lambda_2(u + c)^2 + \dfrac{1}{2}\lambda_3(u - c)^2
\end{bmatrix}
\end{align}

\section{TVD格式}
在本节中,我们以单个拟线性双曲方程为例,推导TVD格式。对于方程组的情形,只需将限制器分别作用于各个分量即可。

对于如下方程:
\begin{align}
    u_t + f(u)_x = 0 
\end{align}

利用FVS可以将通量分解为两部分$f^p$和$f^n$。其中$f^p$对应的界面处数值通量有如下三种可选计算格式:
\begin{align}
    \text{一阶迎风:} \qquad &\widehat{f}^p_{j+\frac{1}{2}} = f^p_{j} \\
    \text{二阶中心:} \qquad &\widehat{f}^p_{j+\frac{1}{2}} = f^p_{j} + \frac{1}{2}(f^p_{j+1}-f^p_j)\\
    \text{二阶迎风:} \qquad &\widehat{f}^p_{j+\frac{1}{2}} = f^p_{j} + \frac{1}{2}(f^p_{j}-f^p_{j-1})\\
\end{align}

TVD格式的基本思想就是通过引入限制器,在光滑区选择精度更高的二阶格式,在间断部分选择一阶格式(
一阶迎风格式的强耗散性可以抑制震荡)。同时通过保证半离散情形下$u_i(t)$总变差随时间不增,进一步抑制震荡。

为此,将上式改写为:
\begin{align}
    \widehat{f}^p_{j+\frac{1}{2}} = f^p_{j} + \frac{1}{2}\phi(r^p_j)(f^p_{j+1}-f^p_{j})
\end{align}

其中,$r^p_j = \frac{f^p_j-f^p_{j-1}}{f^p_{j+1}-f^p_{j}}$。$\phi$的取值对应不同的差分格式:
\begin{align}
    \phi (r) = r,&\quad\text{二阶迎风}\\
    \phi (r) = 1,&\quad\text{二阶中心}\\
    \phi (r) = 0,&\quad\text{一阶中心}
\end{align}

对于限制器$\phi$的限制规则如下:
\begin{enumerate}
    \item $\phi\geqslant0$。
    \item $\phi$在$r\leqslant0$时取0。这是因为$r$小于等于0时可以认为出现了数值振荡,需要用低阶格式抑制震荡。这也带来了TVD格式在极值点处精度下降的问题。
    \item $\phi$在$r\geqslant0$时取值应在$r$和$1$之间。
\end{enumerate}

进一步,有如下计算结果:
\begin{align}
    \frac{\mathrm{d}{u_j}}{\mathrm{d}{t}} &= -\frac{1}{\Delta x}(\widehat{f}^p_{j+\frac{1}{2}} - \widehat{f}^p_{j-\frac{1}{2}})\\
    &=-\frac{1}{2\Delta x}(\frac{\phi(r^p_j)}{r^p_j} - \phi{r^p_{j+1}} + 2)\frac{f^p_j - f^p_{j-1}}{u_j-u_{j-1}}(u_j-u_{j-1})
\end{align}

其中$\frac{f^p_j - f^p_{j-1}}{u_j-u_{j-1}}$对于正通量为正,故由三点半离散TVD格式条件,有:
\begin{align}
    \frac{\phi(r^p_j)}{r^p_j} - \phi(r^p_{j+1}) + 2 > 0 \Rightarrow \phi(r) < 2, \phi(r) < 2r
\end{align}

上述条件相当于给出了$\phi$允许的取值范围。在本次作业中,使用如下三种限制器进行数值测试:
\begin{enumerate}
    \item \textbf{Van Leer 限制器} 
    \begin{equation*}
        \phi(r) = \frac{r + |r|}{1 + |r|} = 
        \begin{cases}
            0 & \text{如果 } r \leq 0, \\
            \dfrac{2r}{1 + r} & \text{如果 } r > 0.
        \end{cases}
    \end{equation*}

    \item \textbf{Minmod 限制器}
    \begin{equation*}
        \phi(r) = \text{minmod}(1, r) = 
        \begin{cases}
            0 & \text{如果 } r \leq 0, \\
            r & \text{如果 } 0 < r \leq 1, \\
            1 & \text{如果 } r > 1.
        \end{cases}
    \end{equation*}

    \item \textbf{Superbee 限制器}
    \begin{equation*}
        \phi(r) = \max\left(0, \min(1, 2r), \min(2, r)\right) = 
        \begin{cases}
            0 & \text{如果 } r \leq 0, \\
            2r & \text{如果 } 0 < r \leq 0.5, \\
            1 & \text{如果 } 0.5 < r \leq 1, \\
            r & \text{如果 } 1 < r \leq 2, \\
            2 & \text{如果 } r > 2.
        \end{cases}
    \end{equation*}
\end{enumerate}

其中Superbee限制器相当于取$\phi$可取值的上界,Minmod相当于下界,Van Leer介于二者之间。

对于负通量,采用类似的方法可以得到对应的数值通量计算格式(迎风格式需要采用最多到右侧两个节点),这里不再赘述。注意限制器可以采用完全一致的形式,
只是自变量形式不同。

\section{NND格式}
我们从单波方程(a>0)出发,再拓展至单个守恒律方程。守恒律方程组的情形是单个方程情形的自然推广。

空间项二阶差分后的修正方程为:
\begin{align}
    u_t + au_x = -\frac{a\Delta x^2}{6}u_{xxx}+Res.&\quad \text{二阶中心差分}\\
    u_t + au_x = \frac{a\Delta x^2}{3}u_{xxx}+Res.&\quad \text{二阶迎风差分}
\end{align}

对于NND一类的频散控制格式,我们认为间断左侧空间一阶导数与二阶导数的乘积为正,即色散项系数为正;间断右侧为负,色散项系数为负。
于是,对于a>0的情况,我们在间断左侧采用迎风格式,在间断右侧采用中心差分格式,在a<0时反之。推广此做法至单个守恒律方程,我们有:
\begin{align}
    &\widehat{f}^p_{j+\frac{1}{2}} =
    \begin{cases}
        f^p_j + \frac{1}{2}(f^p_j-f^p_{j-1})  \qquad \text{间断左侧} \\
        f^p_j + \frac{1}{2}(f^p_{j+1} - f^p_{j}) \qquad \text{间断右侧}
    \end{cases} \\
    &\widehat{f}^n_{j+\frac{1}{2}} =
    \begin{cases}
        f^n_{j+1} - \frac{1}{2}(f^n_{j+1} - f^n_{j}) \qquad \text{间断左侧} \\
        f^n_{j+1} - \frac{1}{2}(f^n_{j+2} - f^n_{j+1}) \qquad \text{间断右侧}
    \end{cases}
\end{align}

下面讨论间断左右侧的判断方法。以$f^p$为例,在间断附近(假定间断由于耗散是有一定厚度的),$f^p$的导数可以视为大于0的常数,于是有:
\begin{align}
    f^p_{j+1}-f^p_{j} = a(u_{j+1}-u_{j}) \\
    f^p_{j}-f^p_{j-1} = a(u_{j}-u_{j-1}) \\
\end{align}

在间断左侧,$u_{j+1}-u_{j}$与$u_j-u_{j-1}$同号,且$|u_{j+1} - u_j| > |u_j-u_{j-1}|$,于是对于正通量也有对应的关系。
在间断右侧,有保持同号,大小相反的关系。进一步的,若二者反号,则视为震荡(这与TVD格式的思路一致,事实上,NND格式与TVD采用Minmod限制器的格式完全一致)。

基于此,给出界面数值通量的计算公式:
\begin{align}
    \widehat{f}^p_{j+\frac{1}{2}} &= f^p_j + \frac{1}{2}Minmod(f^p_{j+1}-f^p_{j}, f^p_{j}-f^p_{j-1})\\
    \widehat{f}^n_{j+\frac{1}{2}} &= f^n_{j+1} - \frac{1}{2}Minmod(f^n_{j+1}-f^n_{j}, f^n_{j+2}-f^n_{j+1})
\end{align}

\section{五阶WENO格式}
仍以单个守恒律方程为例。对于$f^p$,我们以$j-3, j-2, j-1, j, j, j+1, j+2$为基架点,分别考察从左到右三组每组各四个点的三阶格式,有:
\begin{align}
    (\widehat{f}^p_{j+\frac{1}{2}})^0 &= \frac{11}{6}f^p_j - \frac{7}{6}f^p_{j-1} + \frac{1}{3}f^p_{j-2} \\
    (\widehat{f}^p_{j+\frac{1}{2}})^1 &= \frac{1}{3}f^p_{j+1} + \frac{5}{6}f^p_{j} - \frac{1}{6}f^p_{j-1} \\
    (\widehat{f}^p_{j+\frac{1}{2}})^2 &= -\frac{1}{6}f^p_{j+2} - \frac{5}{6}f^p_{j+1} + \frac{1}{3}f^p_{j} \\
\end{align}

以上三种格式均为三阶精度,差分所得导数也为三阶精度。如果利用全部六个节点,得到的数值通量可以如下表示:
\begin{align}
    \widehat{f}^p_{j+\frac{1}{2}} = \frac{1}{10}(\widehat{f}^p_{j+\frac{1}{2}})^0 + \frac{3}{5}(\widehat{f}^p_{j+\frac{1}{2}})^1 + \frac{3}{10}(\widehat{f}^p_{j+\frac{1}{2}})^2
\end{align}

此时得到的数值通量是五阶精度,差分得到的导数也是五阶精度。如果记上式的三个原始权重为$C^0, C^1, C^2$, WENO格式的思路就是为每个三节点计算得到的数值通量分配权重,尽可能让处于光滑区的数值通量的权重接近原始权重,而处于
间断区的数值通量尽可能减少权重。为此,引入光滑度量$\beta^i$,用于衡量每组三节点区间的光滑性。同时记$\alpha^i=\frac{C^i}{(\epsilon+\beta^i)^2}$($\epsilon$是一个小量,避免分母为0),给出数值通量的加权计算式如下:
\begin{align}
    \widehat{f}^p_{j+\frac{1}{2}} = \frac{\alpha^0(\widehat{f}^p_{j+\frac{1}{2}})^0 + \alpha^1(\widehat{f}^p_{j+\frac{1}{2}})^1 + \alpha^2(\widehat{f}^p_{j+\frac{1}{2}})^2}{\sum \alpha^i}
\end{align}

这样,如果在光滑区$\beta^i=O(\Delta x^2)$,在间断区$\beta^i=O(1)$。由于间断区占比很少,权重的分母为$O(\Delta x^{-4})$量阶,对于光滑区权重变为$O(1)$,非光滑区权重变为$O(\Delta x^4)$,从而达到前述的目的。

一个关键的问题是$\beta^i$该如何选取。一个笔者一开始使用的启发性的给法是:
\begin{align}
    \beta^i = &(f^p_{j+i} - 2f^p_{j+i-1}+f^p_{j+i-2})^2+\frac{1}{4}(3f^p_{j+i}-4f^p_{j+i-1}+f^p_{j+i-2})^2 \\
    &+ \frac{1}{4}(f^p_{j+i} - f^p_{j+i-2})^2+\frac{1}{4}(f^p_{j+i}-4f^p_{j+i-1}+3f^p_{j+i-2})^2
\end{align}
给定这一系数的思路是计算所有三节点区间上的二阶精度导数的平方和。

然而,在Jiang and Shu et al.中,作者给出了使得加权以后差分格式达到五阶精度的方法,其给法不同于上述。首先,如果最终的权重在光滑区能写成$C^i+O(\Delta x^2)$的形式,
由前述各差分格式的精度,通过Taylor展开不难证明在所有基架均光滑的情况下最后计算出的通量导数是五阶精度的。
而若想达到此条件,需对$\beta^i$做出限制。若$\beta^i=D(1+O(\Delta x^2))$,其中D是一个对所有当前光滑度量均相等的常数,
则:
\begin{align}
    &\alpha^i = \frac{C^i}{D^2(1+O(\Delta x^2))^2}=\frac{C^i}{D^2}(1-2O(\Delta x^2)) \\
    \Rightarrow &\alpha^i = \frac{\frac{C^i}{D^2}(1+O(\Delta x^2))}{\frac{1}{D^2}(1+O(\Delta x^2))} = C^i + O(\Delta x^2)
\end{align}
可以验证,对于上述提到的光滑度量给法,Taylar展开后只能写为$D(1+O(\Delta x))$;进一步,如果j点是极值点,一阶导数为0,则这样的光滑度量甚至不能写为此种形式。
换句话说,上述光滑度量的精度阶数并不稳定。

而对于论文中给出的WENO5光滑度量的经典形式,展开为:
\begin{align}
    \beta^0 = \frac{13}{12}(f''\Delta x^2)^2+\frac{1}{4}(2f'\Delta x-\frac{2}{3}f'''\Delta x^3)^2 + O(\Delta x^6)\\
    \beta^1 = \frac{13}{12}(f''\Delta x^2)^2+\frac{1}{4}(2f'\Delta x+\frac{1}{3}f'''\Delta x^3)^2 + O(\Delta x^6)\\
    \beta^2 = \frac{13}{12}(f''\Delta x^2)^2+\frac{1}{4}(2f'\Delta x-\frac{2}{3}f'''\Delta x^3)^2 + O(\Delta x^6)
\end{align}
可以发现无论是一般光滑情况还是极值点情况,上述光滑度量均可写为$D(1+O(\Delta x^2))$,因此可以保证五阶精度。

对比两种格式,前者对于不光滑区域更敏感,但精度飘动;后者虽然捕捉间断的能力稍低,但精度稳定。值得思考的是,在Sod激波管问题中,二者表现相近。这一现象将在结果解释中进行一定分析。

对于负通量,注意基架点应选对应的迎风方向的六个点。这里不再赘述。

\section{Roe格式}
在$[j, j+1]$上,将Euler方程组近似为$\mathbf{U}_t+\widehat{\mathbf{A}}(\mathbf{U}_L, \mathbf{U}_R)\mathbf{U}_x=0$。其中$\mathbf{U}_L$和$\mathbf{U}_R$分别为该区间上的左右状态,$\widehat{\mathbf{A}}$
称为Roe矩阵,需满足如下性质:
\begin{enumerate}
    \item \textbf{相容性}
    \begin{align}
        \widehat{\mathbf{A}}(\mathbf{U}, \mathbf{U}) = \mathbf{A}(\mathbf{U})
    \end{align}
    \item \textbf{守恒性}
    \begin{align}
        \widehat{\mathbf{A}}(\mathbf{U}_L, \mathbf{U}_R)(\mathbf{U}_R - \mathbf{U}_L)=\mathbf{F}_R-\mathbf{F}_L
    \end{align}
    \item \textbf{双曲性}
    \begin{align}
        \widehat{\mathbf{A}} = \mathbf{R}\bm{\Lambda}\mathbf{R}^{-1}
    \end{align}
\end{enumerate}

我们先讨论如何利用左右状态构造Roe矩阵。定义如下中间变量$\mathbf{z}$,并将$\mathbf{U}$和$\mathbf{F}$写为中间变量的函数形式:
\begin{align}
    \mathbf{z} &= \sqrt{\rho}
    \begin{bmatrix}
        1 \\
        u \\
        H \\
    \end{bmatrix} \\
    H &= \frac{\gamma p}{\rho (\gamma-1)} + \frac{1}{2}u^2 \\
    \mathbf{U} &= 
    \begin{bmatrix}
        z_1^2\\
        z_1z_2\\
        \frac{z_1z_3}{\gamma} + \frac{\gamma - 1}{2\gamma}z_2^2
    \end{bmatrix}\\
    \mathbf{F} &=
    \begin{bmatrix}
        z_1z_2\\
        z_2^2 + \frac{\gamma-1}{\gamma}(z_1z_3-\frac{1}{2}z_2^2)\\
        z_2z_3
    \end{bmatrix}
\end{align}

分别记$\mathbf{U}$和$\mathbf{F}$对$\mathbf{z}$的Jacobi矩阵为$\mathbf{B}$和$\mathbf{C}$。由于通量和守恒量均为中间变量的二次函数,由Taylor展开,有:
\begin{align}
    \Delta\mathbf{U}&=\mathbf{B}(\frac{\mathbf{z}_L+\mathbf{z}_R}{2})\Delta\mathbf{z}\\
    \Delta\mathbf{F}&=\mathbf{C}(\frac{\mathbf{z}_L+\mathbf{z}_R}{2})\Delta\mathbf{z}
\end{align}

于是推出$\widehat{\mathbf{A}}(\mathbf{U}_L, \mathbf{U}_R) = \mathbf{C}(\frac{\mathbf{z}_L+\mathbf{z}_R}{2})\mathbf{B}^{-1}(\frac{\mathbf{z}_L+\mathbf{z}_R}{2})$。

进一步计算可以知道,该矩阵就是:
\begin{align}
    &\begin{bmatrix}
        0&1&0\\
        \frac{\gamma-3}{3}\widetilde{u}^2&(3-\gamma)\widetilde{u}&\gamma-1\\
        -\widetilde{u}\widetilde{H}+\frac{1}{2}(\gamma-1)\widetilde{u}^3&\widetilde{H}-(\gamma-1)\widetilde{u}^2&\gamma\widetilde{u} 
    \end{bmatrix}\\
    &\widetilde{u} = \frac{\sqrt{\rho_L}u_L+\sqrt{\rho_R}u_R}{\sqrt{\rho_L}+\sqrt{\rho_R}},\qquad\widetilde{H} = \frac{\sqrt{\rho_L}H_L+\sqrt{\rho_R}H_R}{\sqrt{\rho_L}+\sqrt{\rho_R}}
\end{align}

由于其与Jacobi矩阵具有完全相同的形式,知其必满足双曲性和相容性。

利用该矩阵的分解$\widehat{\mathbf{A}} = \mathbf{R}\bm{\Lambda}\mathbf{R}^{-1}$,记$\mathbf{V}=\mathbf{R}^{-1}\mathbf{U}$,则方程解耦为$V^i_t+\lambda^iV^i_x=0$。
对应有$\mathbf{V}_L=\mathbf{R}^{-1}\mathbf{U}_L$,$\mathbf{V}_R=\mathbf{R}^{-1}\mathbf{U}_R$。

此时,由波速的正负,$V^i_{j+\frac{1}{2}}$应分别取左状态或右状态,即:
\begin{align}
    V^i_{j+\frac{1}{2}} = \frac{1}{2}(V^i_L+V^i_R) - \frac{1}{2}sgn(\lambda^i)(V^i_R-V^i_L)
\end{align}
利用Roe矩阵,进而可以计算出界面通量:
\begin{align}
    \mathbf{F}_{j+\frac{1}{2}} &= \frac{1}{2}(\mathbf{F}_L+\mathbf{F}_R)-\frac{1}{2}|\widehat{\mathbf{A}}|(\mathbf{U}_R-\mathbf{U}_L)\\
    |\widehat{\mathbf{A}}| &= \mathbf{R}|\bm{\Lambda}|\mathbf{R}^{-1}
\end{align}

最后$|\widehat{\mathbf{A}}|$的计算可以通过解析式算出,无需求逆。在本次作业中,左状态和右状态直接取为节点上的值。

\section{特征重构}
特征重构法就是将原先的数值通量改换到特征空间计算,计算完成后再返回物理空间处理。以WENO格式为例,计算$j+\frac{1}{2}$处的正负数值通量需要用到$[j-2,j+3]$共六个节点处的正或负通量。若使用特征重构法,先利用$j+\frac{1}{2}$处通量的某种平均值计算出Jacobi矩阵$\mathbf{A}$在处的特征矩阵$\mathbf{R}_{j+\frac{1}{2}}$及其逆(均可解析计算),
再将所有要用到的节点处通量左乘$\mathbf{R}_{j+\frac{1}{2}}^{-1}$,并将此新得到的特征空间中通量按WENO格式加权处理。最终再将得到的$j+\frac{1}{2}$处特征空间中的数值通量左乘特征矩阵$\mathbf{R}_{j+\frac{1}{2}}$返回到物理空间中,并以此计算导数。其他格式的特征重构类似。

\section{时间推进的三阶Runge-Kutta和时间步长的确定}
在半离散格式下,时间推进计算即数值求解以各个节点守恒量为变量的自治常微分方程组。在本节,我们首先给出自治系统的三阶Runge-Kutta推导,再讨论时间步长如何动态调整。

对于自治系统$\frac{\mathrm{d}\mathbf{U}}{\mathrm{d}t}=\mathbf{R}(\mathbf{U})$,设时间步长为$h$,$\mathbf{U}^{n+1} = \mathbf{U}^n + h(b_1\mathbf{k}_1+b_2\mathbf{k}_2+b_3\mathbf{k}_3)$。
其中:
\begin{align}
    \mathbf{k}_1 &= \mathbf{R}(\mathbf{U}^n)\\
    \mathbf{k}_2 &= \mathbf{R}(\mathbf{U}^n+ha_{21}\mathbf{k}_1)\\
    \mathbf{k}_3 &= \mathbf{R}(\mathbf{U}^n + h(a_{31}\mathbf{k}_1+a_{32}\mathbf{k}_2))
\end{align}

将$\mathbf{U}^{n+1}$在$\mathbf{U}^{n}$处Taylor展开,得:
\begin{align}
    \mathbf{U}^{n+1} = \mathbf{U}^n + \mathbf{R}h + (\mathbf{R}\nabla\cdot\mathbf{R})\frac{h^2}{2}+(\mathbf{R}\nabla\nabla\cdot\mathbf{R}\cdot\mathbf{R}+\mathbf{R}\nabla\cdot\mathbf{R}\nabla\cdot\mathbf{R})\frac{h^3}{6} + &O(h^4)
\end{align}

再展开$\mathbf{k}_2$和$\mathbf{k}_3$:
\begin{align}
    \mathbf{k}_2 &=  \mathbf{R} + \mathbf{R}\nabla\cdot\mathbf{R}a_{21}h + (\mathbf{R}\nabla\nabla\cdot\mathbf{R}\cdot\mathbf{R})\frac{a_{21}^2h^2}{2}+O(h^3)\\
    \mathbf{k}_3 &= \mathbf{R} + \mathbf{R}\nabla\cdot\mathbf{R}(a_{31}+a_{32})h + \mathbf{R}\nabla\cdot\mathbf{R}\nabla\cdot\mathbf{R}a_{21}a_{32}h^2\\
    &+\mathbf{R}\nabla\nabla\cdot\mathbf{R}\cdot\mathbf{R}(a_{21}+a_{32})^2\frac{h^2}{2}+O(h^3)
\end{align}

代回$\mathbf{U}^{n+1}$预设的表达式,并与Taylor展开的结果比对系数,得到:
\begin{align}
    \begin{cases}
        b_!+b_2+b_3=1\\
        a_{31}b_3+a_32b_3+a_21b_2=\frac{1}{2}\\
        a_{21}^2b_{2}+b_{3}(a_{31}+a_{32})^2 = \frac{1}{3}\\
        a_{21}a_{32}b_3=\frac{1}{6}
    \end{cases}
\end{align}

在本次作业中,采用Heun格式,即令:
\begin{align}
    \begin{cases}
        b_1=\frac{1}{4}\quad b_2=0 \quad b_3=\frac{3}{4}\\
        a_{21} = \frac{1}{3}\\
        a_{31}=0\quad a_{32} = \frac{2}{3}
    \end{cases}
\end{align}

对于时间步长的选择,为保持计算稳定有如下限制:
\begin{align}
    \Delta t \leqslant\mathbf{CFL}\frac{\Delta x}{|u_{max}|+c}
\end{align}

需对每次时间推进按此式计算时间步长。
\end{document}