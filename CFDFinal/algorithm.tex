\documentclass[12pt, a4paper]{article}
\usepackage{ctex} % 支持中文处理
\usepackage{geometry} % 页面布局
\usepackage{graphicx} % 图片支持
\usepackage{hyperref} % 超链接支持
\usepackage{amsmath} % 数学公式
\usepackage{amsfonts}
\usepackage{amssymb}
\usepackage{amsthm}
\usepackage{bm}
\usepackage{color}
\usepackage{physics}
\newtheorem{lemma}{引理}
\newtheorem{theorem}{定理}
\geometry{left=2.5cm,right=2.5cm,top=2.5cm,bottom=2.5cm} % 设置页边距
\title{数理算法原理}
\author{安庭毅\ 工学院 \ 2100011014}
\date{\today} % 使用今天的日期

\begin{document}

\maketitle % 显示标题
\section{FVS通量分裂}
在讨论欧拉方程组的半离散差分格式时,若使用迎风格式,
需要根据波传播的特征方向采用不同的基点计算数值通量。因此,再开始计算之前,可以利用FVS方法将通量按一定方式分裂为正负方向传播的通量。
在本次作业中,我们统一使用Steger-Warming通量分裂方法,其基本思路如下:

通量$F(U)$可以分解为$A(U)U$,其中$A(U)$是通量对守恒变量的Jacobi矩阵。由双曲性,$A$可以对角化,记$A=R\Lambda R^{-1}$。将$\Lambda$分解为正负部分,并以小量
修正保证分裂后矩阵的光滑性,即:
\begin{align}
    \lambda^{\pm} = \frac{\lambda\pm\sqrt{\lambda ^ 2 + \epsilon ^ 2}}{2} 
\end{align}

再计算$F\pm=R\Lambda^{\pm}R^{-1}U$即得到分裂后沿不同方向的通量。之后分别利用这两个通量按对应的差分格式计算界面数值通量即可。
\section{TVD格式}
\subsection{数值通量计算}
在本节中,我们以单个拟线性双曲方程为例,推导TVD格式。对于方程组的情形,只需将限制器分别作用于各个分量即可。

对于如下方程:
\begin{align}
    u_t + f(u)_x = 0 
\end{align}
其中通量的差分有如下三种可选格式
\end{document}