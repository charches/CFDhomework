\documentclass[12pt, a4paper]{article}
\usepackage{ctex} % 支持中文处理
\usepackage{geometry} % 页面布局
\usepackage{graphicx} % 图片支持
\usepackage{hyperref} % 超链接支持
\usepackage{amsmath} % 数学公式
\usepackage{amsfonts}
\usepackage{amssymb}
\usepackage{amsthm}
\usepackage{bm}
\usepackage{color}
\usepackage{physics}
\newtheorem{lemma}{引理}
\newtheorem{theorem}{定理}
\geometry{left=2.5cm,right=2.5cm,top=2.5cm,bottom=2.5cm} % 设置页边距
\title{数理算法原理}
\author{安庭毅\ 工学院 \ 2100011014}
\date{\today} % 使用今天的日期

\begin{document}

\maketitle % 显示标题
\section{FVS通量分裂}
在讨论欧拉方程组的半离散差分格式时,若使用迎风格式,
需要根据波传播的特征方向采用不同的基点计算数值通量。因此,在开始计算之前,可以利用FVS方法将通量按一定方式分裂为正负方向传播的通量。
在本次作业中,我们统一使用Steger-Warming通量分裂方法,其基本思路如下:

通量$F(U)$可以分解为$A(U)U$,其中$A(U)$是通量对守恒变量的Jacobi矩阵。由双曲性,$A$可以对角化,记$A=R\Lambda R^{-1}$。将$\Lambda$分解为正负部分,并以小量
修正保证分裂后矩阵的光滑性,即:
\begin{align}
    \lambda^{\pm} = \frac{\lambda\pm\sqrt{\lambda ^ 2 + \epsilon ^ 2}}{2} 
\end{align}
再计算$F\pm=R\Lambda^{\pm}R^{-1}U$即得到分裂后沿不同方向的通量,分别利用这两个通量按对应的差分格式计算界面数值通量即可。
由于Jacobi矩阵的对角化可以解析算出,所以无需数值求解对角化。对于$\Lambda$是对角线元素为$u, u+c, u-c$处理之后的对角矩阵,对应的通量可由如下公式计算:

\begin{align}
\mathbf{F} = \frac{\rho}{2\gamma}
\begin{bmatrix}
2(\gamma - 1)\lambda_1 + \lambda_2 + \lambda_3 \\
2(\gamma - 1)\lambda_1 u + (u + c)\lambda_2 + (u - c)\lambda_3 \\
(\gamma - 1)\lambda_1 u^2 + \dfrac{3 - \gamma}{2(\gamma - 1)}(\lambda_2 + \lambda_3)c^2 + \dfrac{1}{2}\lambda_2(u + c)^2 + \dfrac{1}{2}\lambda_3(u - c)^2
\end{bmatrix}
\end{align}

\section{TVD格式}
在本节中,我们以单个拟线性双曲方程为例,推导TVD格式。对于方程组的情形,只需将限制器分别作用于各个分量即可。

对于如下方程:
\begin{align}
    u_t + f(u)_x = 0 
\end{align}

利用FVS可以将通量分解为两部分$f^p$和$f^n$。其中$f^p$对应的界面处数值通量有如下三种可选计算格式:
\begin{align}
    \text{一阶迎风:} \qquad &\widehat{f}^p_{j+\frac{1}{2}} = f^p_{j} \\
    \text{二阶中心:} \qquad &\widehat{f}^p_{j+\frac{1}{2}} = f^p_{j} + \frac{1}{2}(f^p_{j+1}-f^p_j)\\
    \text{二阶迎风:} \qquad &\widehat{f}^p_{j+\frac{1}{2}} = f^p_{j} + \frac{1}{2}(f^p_{j}-f^p_{j-1})\\
\end{align}

TVD格式的基本思想就是通过引入限制器,在光滑区选择精度更高的二阶格式,在间断部分选择一阶格式(
一阶迎风格式的强耗散性可以抑制震荡)。同时通过保证半离散情形下$u_i(t)$总变差随时间不增,进一步抑制震荡。

为此,将上式改写为:
\begin{align}
    \widehat{f}^p_{j+\frac{1}{2}} = f^p_{j} + \frac{1}{2}\phi(r^p_j)(f^p_{j+1}-f^p_{j})
\end{align}

其中,$r^p_j = \frac{f^p_j-f^p_{j-1}}{f^p_{j+1}-f^p_{j}}$。$\phi$的取值对应不同的差分格式:
\begin{align}
    \phi (r) = r,&\quad\text{二阶迎风}\\
    \phi (r) = 1,&\quad\text{二阶中心}\\
    \phi (r) = 0,&\quad\text{一阶中心}
\end{align}

对于限制器$\phi$的限制规则如下:
\begin{enumerate}
    \item $\phi\geqslant0$。
    \item $\phi$在$r\leqslant0$时取0。这是因为$r$小于等于0时可以认为出现了数值振荡,需要用低阶格式抑制震荡。这也带来了TVD格式在极值点处精度下降的问题。
    \item $\phi$在$r\geqslant0$时取值应在$r$和$1$之间。
\end{enumerate}

进一步,有如下计算结果:
\begin{align}
    \frac{\mathrm{d}{u_j}}{\mathrm{d}{t}} &= -\frac{1}{\Delta x}(\widehat{f}^p_{j+\frac{1}{2}} - \widehat{f}^p_{j-\frac{1}{2}})\\
    &=-\frac{1}{2\Delta x}(\frac{\phi(r^p_j)}{r^p_j} - \phi{r^p_{j+1}} + 2)\frac{f^p_j - f^p_{j-1}}{u_j-u_{j-1}}(u_j-u_{j-1})
\end{align}

其中$\frac{f^p_j - f^p_{j-1}}{u_j-u_{j-1}}$对于正通量为正,故由三点半离散TVD格式条件,有:
\begin{align}
    \frac{\phi(r^p_j)}{r^p_j} - \phi(r^p_{j+1}) + 2 > 0 \Rightarrow \phi(r) < 2, \phi(r) < 2r
\end{align}

上述条件相当于给出了$\phi$允许的取值范围。在本次作业中,使用如下三种限制器进行数值测试:
\begin{enumerate}
    \item \textbf{Van Leer 限制器} 
    \begin{equation*}
        \phi(r) = \frac{r + |r|}{1 + |r|} = 
        \begin{cases}
            0 & \text{如果 } r \leq 0, \\
            \dfrac{2r}{1 + r} & \text{如果 } r > 0.
        \end{cases}
    \end{equation*}

    \item \textbf{Minmod 限制器}
    \begin{equation*}
        \phi(r) = \text{minmod}(1, r) = 
        \begin{cases}
            0 & \text{如果 } r \leq 0, \\
            r & \text{如果 } 0 < r \leq 1, \\
            1 & \text{如果 } r > 1.
        \end{cases}
    \end{equation*}

    \item \textbf{Superbee 限制器}
    \begin{equation*}
        \phi(r) = \max\left(0, \min(1, 2r), \min(2, r)\right) = 
        \begin{cases}
            0 & \text{如果 } r \leq 0, \\
            2r & \text{如果 } 0 < r \leq 0.5, \\
            1 & \text{如果 } 0.5 < r \leq 1, \\
            r & \text{如果 } 1 < r \leq 2, \\
            2 & \text{如果 } r > 2.
        \end{cases}
    \end{equation*}
\end{enumerate}

其中Superbee限制器相当于取$\phi$可取值的上界,Minmod相当于下界,Van Leer介于二者之间。

对于负通量,采用类似的方法可以得到对应的数值通量计算格式(迎风格式需要采用最多到右侧两个节点),这里不再赘述。注意限制器可以采用完全一致的形式,
只是自变量形式不同。
\end{document}