\documentclass[12pt, a4paper]{article}
\usepackage{ctex} % 支持中文处理
\usepackage{geometry} % 页面布局
\usepackage{graphicx} % 图片支持
\usepackage{hyperref} % 超链接支持
\usepackage{amsmath} % 数学公式
\usepackage{amsfonts}
\usepackage{amssymb}
\usepackage{amsthm}
\usepackage{bm}
\usepackage{color}
\usepackage{physics}

\geometry{left=2.5cm,right=2.5cm,top=2.5cm,bottom=2.5cm} % 设置页边距
\title{数理算法原理}
\author{安庭毅\ 工学院 \ 2100011014}
\date{\today} % 使用今天的日期

\begin{document}

\maketitle % 显示标题
\section{一阶迎风格式}
对于格式
\begin{align}
  u^{n+1}_{j} = u^{n}_{j} - c(u^n_j - u^n_{j-1})
\end{align}
在周期边界条件的假定下,采用Fourier分析法分析其稳定性。

记延拓函数为
\begin{align}
  u^n(x) = u^n_j, x_j \leqslant x < x_{j+1}
\end{align}
容易知道该延拓函数在$\mathbb{R}$上的$L_2$范数等于第n层数值解的离散$L_2$范数。

则上述迎风格式转化为:
\begin{align}
  u^{n+1}(x) = u^n(x) - c(u^n(x) - u^n(x-h))
\end{align}

两侧做Fourier变换,得到:
\begin{align}
  \hat{u}^{n+1}(k) = (1 - c + ce^{-ikh})\hat{u}^n(k)
\end{align}

两侧取L2范数,由Parseval等式,有
\begin{align}
  \Vert u^n \Vert ^2 = \Vert \hat{u}^n \Vert ^ 2 = \int _\mathbb{R} |1 - c + ce^{-ikh} |^{2n} |\hat{u}^0(k)|^2 \dd{k}
\end{align}

故知稳定性条件为:
\begin{align}
  &|G| = |1 - c + ce^{-ikh}| \leqslant 1 \\
  &\Rightarrow \Vert u^{n} \Vert \leqslant \sqrt{\int _\mathbb{R} |\hat{u}^0(k)|^2 \dd{k}} = \Vert u^0 \Vert
\end{align}

即
\begin{align}
  |c| \leqslant 1
\end{align}

下考虑CFL数固定时的收敛阶。记$(x_j, t_n)$处真实解与数值解之差为$\epsilon^n_j$, 将$\epsilon^n_j$分解为真实解与将真实解代入差分方程得到的
$(x_j, t_n)$处的值的差,和该差与数值解的差,利用Taylor展开,有:
\begin{align}
  \epsilon^{n+1}_j = E^n_j h^2 + (1 - c)\epsilon^n_j + c\epsilon^n_{j-1}
\end{align}
其中$E^n_j$是只和Taylor展开Lagrange余项中的导数相关的项。

利用Fourier分析类似的方法,可得:
\begin{align}
  \Vert\epsilon^{n + 1}\Vert \leqslant \Vert E^n \Vert h^2 + |G|\Vert \epsilon^n \Vert
\end{align}

若认为$\Vert E^n \Vert$在h趋于0时对n是一致有界$L$的,考虑到推进到固定时刻T时,$n=\frac{T}{\Delta t}=\frac{aT}{ch}=O(\frac{1}{h})$,
并且格式稳定时$|G|\leqslant1$,有:
\begin{align}
  \Vert \epsilon^{n} \Vert \leqslant nLh^2 + \Vert \epsilon^{0} \Vert = O(h)
\end{align}
故一阶迎风格式的收敛阶应为1。

最后,由一阶迎风格式的修正方程
\begin{align}
  u_t + au_x = \frac{ah}{2}(1 - c)u_{xx} - \frac{ah^2}{6}(1 - c)(1 - 2c)u_{xxx} + Res
\end{align}
可知数值解以耗散为主导,当时间推进到足够大时,应能观察到振幅的变化。

\section{LaxWendroff格式}
使用上述类似的方法,可得LaxWendroff格式的放大因子为
\begin{align}
  G = 1 - c^2(1 - \cos kh) - ic\sin kh
\end{align}
于是稳定性条件为$|c|\leqslant 1$。

同样,类似的方法可以得到收敛阶为2,这里不再赘述。

考虑修正方程:
\begin{align}
  u_t + au_x = -\frac{ah^2}{6}(1 - c^2)u_{xxx} - \frac{ah^3}{8}c(1 - c^2)u_{xxxx} + Res
\end{align}

可知数值解以色散为主导,随时间推进,应能观察到相位的误差。

\section{蛙跳格式}
对于蛙跳格式
\begin{align}
  u^{n + 1}_{j} = u^{n - 1}_j - c(u^n_{j + 1} - u^n_{j - 1})
\end{align}

将其转写为差分方程组
\begin{align}
  \begin{bmatrix}
    u^{n + 1}_j \\
    v^{n + 1}_j
  \end{bmatrix}
  =
  \begin{bmatrix}
    v_j^n - c(u^n_{j + 1} - u^n_{j - 1}) \\
    u_j^n
  \end{bmatrix}
\end{align}

仿照1中得到延拓函数$\textbf{U}^n(x)$,两侧做Fourier变换,得
\begin{align}
  \hat{\textbf{U}}^{n + 1}(k) = 
  \begin{bmatrix}
    ce^{-ikh} - ce^{ikh}  &1 \\
    1  &0
  \end{bmatrix}
  \hat{\textbf{U}}^n(k)
\end{align}

考虑到此时的增长矩阵$G$是正规矩阵,可以酉正交对角化,记为$\textbf{G} = \textbf{P}^{H}\textbf{D}\textbf{P}$,
则有:
\begin{align}
  \Vert \textbf{U}^n \Vert^2 = \Vert \hat{\textbf{U}}^n \Vert^2 = \int_\mathbb{R} (\textbf{P}\hat{\textbf{U}}^0)^H (\overline{\textbf{D}}\textbf{D})^n (\textbf{P}\hat{\textbf{U}}^0) \dd{k} 
\end{align}

考虑到增长矩阵对k是连续的,故其谱半径是k的连续函数。而$\textbf{D}$的元素即为$\textbf{G}$的特征值,于是由上式可知,格式稳定等价于$\textbf{G}$的谱半径对任意k均小于等于1。即:
\begin{align}
  |G^{\pm}| = |-ic\sin kh \pm (1 - c^2\sin^2kh)| \leqslant 1 \Leftrightarrow |c| \leqslant 1
\end{align}

故格式稳定要求c的模长不大于1。

同样,进行类似于1中的分析可知此时收敛阶为2,这里不再赘述。

考虑蛙跳格式的修正方程:
\begin{align}
  u_t + au_x = -\frac{ah^2}{6}(1 - c^2)u_{xxx} - \frac{ah^4}{120}(1-c^2)(1-9c^2)u_{xxxxx} + Res
\end{align}
可以看出蛙跳格式的修正方程无偶数阶导数,所以是无耗散的;由于色散项占主导,随时间推进可观察到相位的误差。
\end{document}
